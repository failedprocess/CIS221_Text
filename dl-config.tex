%*****
% classicthesis-config.tex 
%*****

\PassOptionsToPackage{utf8}{inputenc}
	\usepackage{inputenc}

%*****
% 1. Configure classicthesis
%*****

\PassOptionsToPackage{eulerchapternumbers,listings,%drafting,%
					 pdfspacing,floatperchapter,%linedheaders,%
					 subfig,beramono,eulermath,parts}{classicthesis}                                        
%*****
% Available options for classicthesis.sty 
% (see ClassicThesis.pdf for more information):
% drafting
% parts nochapters linedheaders
% eulerchapternumbers beramono eulermath pdfspacing minionprospacing
% tocaligned dottedtoc manychapters
% listings floatperchapter subfig
%*****

%*****
% 2. Personal data and user ad-hoc commands
%*****
\newcommand{\myTitle}{Exploring Digital Logic\xspace}
\newcommand{\mySubtitle}{With Logisim-Evolution}
\newcommand{\myDegree}{}
\newcommand{\myName}{George Self\xspace}
\newcommand{\myProf}{}
\newcommand{\myOtherProf}{}
\newcommand{\mySupervisor}{}
\newcommand{\myFaculty}{}
\newcommand{\myDepartment}{Computer Information Systems\xspace}
\newcommand{\myUni}{Cochise College\xspace}
\newcommand{\myLocation}{Arizona\xspace}
\newcommand{\myTime}{July 2018\xspace}
\newcommand{\myVersion}{Edition 6.0\xspace}

%*****
% Setup, finetuning, and useful commands
%*****
\newcounter{dummy} % necessary for correct hyperlinks (to index, bib, etc.)
\newlength{\abcd} % for ab..z string length calculation
\providecommand{\mLyX}{L\kern-.1667em\lower.25em\hbox{Y}\kern-.125emX\@}
\newcommand{\ie}{i.\,e.}
\newcommand{\Ie}{I.\,e.}
\newcommand{\eg}{e.\,g.}
\newcommand{\Eg}{E.\,g.}
\newcommand{\blankpage}{ % Create a blank page at the end of the document
  \newpage
  \thispagestyle{empty}
  \mbox{}
  \newpage
  }
% The following creates a function named maxwidth that permits
% me to set a maximum width for images. If the natural width of
% the image is less than maxwidth then the image is rendered at
% its natural size, else scaled to maxwidth.
\makeatletter
\def\maxwidth#1{\ifdim\Gin@nat@width>#1 #1\else\Gin@nat@width\fi}
\makeatother

\newcommand{\Le}{\textit{Logisim-evolution }}


%*****
% My packages
%*****
\usepackage[
type={CC},
modifier={by-nc-sa},
version={4.0}
]{doclicense} % Prints the Creative Commons license

\usepackage[obeyspaces,hyphens]{url}
\expandafter\def\expandafter\UrlBreaks\expandafter{\UrlBreaks%  save the current one
	\do\a\do\b\do\c\do\d\do\e\do\f\do\g\do\h\do\i\do\j%
	\do\k\do\l\do\m\do\n\do\o\do\p\do\q\do\r\do\s\do\t%
	\do\u\do\v\do\w\do\x\do\y\do\z\do\A\do\B\do\C\do\D%
	\do\E\do\F\do\G\do\H\do\I\do\J\do\K\do\L\do\M\do\N%
	\do\O\do\P\do\Q\do\R\do\S\do\T\do\U\do\V\do\W\do\X%
	\do\Y\do\Z}


%*****
% Packages with options that might require adjustments
%*****

% sets up various typographic and hyphenation rules
\PassOptionsToPackage{american}{babel}
\usepackage{babel}                  

% manage both inline and block quotes
\usepackage{csquotes}

% I don't use a bibliography 
%\PassOptionsToPackage{%
%	backend=biber, %instead of bibtex
%	backend=bibtex8,bibencoding=ascii,%
%	language=auto,%
%	style=numeric-comp,%
%	%style=authoryear-comp, % Author 1999, 2010
%	%bibstyle=authoryear,dashed=false, % dashed: substitute rep. author with ---
%	sorting=nyt, % name, year, title
%	maxbibnames=10, % default: 3, et al.
%	%backref=true,%
%	natbib=true % natbib compatibility mode (\citep and \citet still work)
%}{biblatex}
%\usepackage{biblatex}
%\addbibresource{dlbib}

%*****
% Generally useful packages
%*****
\usepackage{fontawesome} % use the fontawesome icons
\usepackage{textcomp} % fix warning with missing font shapes
\usepackage{scrhack} % fix warnings when using KOMA with listings package
\usepackage{xspace} % to get the spacing after macros right  
\usepackage{mparhack} % get marginpar right
\usepackage{fixltx2e} % fixes some LaTeX stuff 
\usepackage[paperheight=11in,paperwidth=8.5in]{geometry}
\usepackage{shorttoc} % generate brief version of the table of contents
\PassOptionsToPackage{fleqn}{amsmath}  % math environments
	\usepackage{amsmath}

% Note: I use the glossaries package for acronymns
%\PassOptionsToPackage{printonlyused,smaller}{acronym} 
%	\usepackage{acronym} % nice macros for handling all acronyms in the thesis
%	\renewcommand*{\aclabelfont}[1]{\acsfont{#1}}

%*****
% 4. Setup floats, tables, (sub)figures, and captions
%*****
\usepackage{tabularx} % better tables
    \setlength{\extrarowheight}{3pt} % increase table row height
\newcommand{\tableheadline}[1]{\multicolumn{1}{c}{\spacedlowsmallcaps{#1}}}
\newcommand{\myfloatalign}{\centering} 
\usepackage{caption}
\captionsetup{font=small} % format=hang,
\usepackage{subfig}  

%*****
% 5. Setup code listings - formatted for Verilog listings
%*****
\usepackage{listings} 
%\lstset{emph={trueIndex,root},emphstyle=\color{BlueViolet}}%\underbar} % for special keywords
\lstset{language=Verilog,%[LaTeX]Tex,
    morekeywords={PassOptionsToPackage,selectlanguage},
    keywordstyle=\color{RoyalBlue},%\bfseries,
    basicstyle=\small\ttfamily,
    %identifierstyle=\color{NavyBlue},
    commentstyle=\color{Green}\ttfamily,
    stringstyle=\rmfamily,
    numbers=left,
    numberstyle=\scriptsize,%\tiny
    stepnumber=2,
    numbersep=8pt,
    showstringspaces=false,
    breaklines=true,
    %frameround=ftff,
    frame=lines,
    captionpos=b,  % put captions at the bottom of the listing
    aboveskip=.75\baselineskip,
    belowskip=.75\baselineskip
    %abovecaptionskip=.75\baselineskip
    %belowcaptionskip=.75\baselineskip
    %frame=L
} 

%*****
% PDFLaTeX
%*****
\PassOptionsToPackage{pdftex,hyperfootnotes=false,pdfpagelabels}{hyperref}
    \usepackage{hyperref}  % backref linktocpage pagebackref
\pdfcompresslevel=9
\pdfadjustspacing=1 
\PassOptionsToPackage{pdftex}{graphicx}
    \usepackage{graphicx} 
 
%*****
% Hyperreferences
%*****

\hypersetup{%
    %draft, % = no hyperlinking at all (useful in b/w printouts)
    colorlinks=true, linktocpage=true, pdfstartpage=3, pdfstartview=FitV,%
    % uncomment the following line if you want to have black links (e.g., for printing)
    %colorlinks=false, linktocpage=false, pdfstartpage=3, pdfstartview=FitV, pdfborder={0 0 0},%
    breaklinks=true, pdfpagemode=UseNone, pageanchor=true, pdfpagemode=UseOutlines,%
    plainpages=false, bookmarksnumbered, bookmarksopen=true, bookmarksopenlevel=1,%
    hypertexnames=true, pdfhighlight=/O,%nesting=true,%frenchlinks,%
    urlcolor=webbrown, linkcolor=RoyalBlue, citecolor=webgreen, %pagecolor=RoyalBlue,%
    %urlcolor=Black, linkcolor=Black, citecolor=Black, %pagecolor=Black,%
    pdftitle={\myTitle},%
    pdfauthor={\textcopyright\ \myName, \myUni, \myFaculty},%
    pdfsubject={},%
    pdfkeywords={},%
    pdfcreator={pdfLaTeX},%
    pdfproducer={LaTeX with hyperref and classicthesis}%
}   

%*****
% Setup autoreferences
%*****

\makeatletter
\@ifpackageloaded{babel}%
    {%
       \addto\extrasamerican{%
			\renewcommand*{\figureautorefname}{Figure}%
			\renewcommand*{\tableautorefname}{Table}%
			\renewcommand*{\partautorefname}{Part}%
			\renewcommand*{\chapterautorefname}{Chapter}%
			\renewcommand*{\sectionautorefname}{Section}%
			\renewcommand*{\subsectionautorefname}{Section}%
			\renewcommand*{\subsubsectionautorefname}{Section}%     
                }%
%       \addto\extrasngerman{% 
%			\renewcommand*{\paragraphautorefname}{Absatz}%
%			\renewcommand*{\subparagraphautorefname}{Unterabsatz}%
%			\renewcommand*{\footnoteautorefname}{Fu\"snote}%
%			\renewcommand*{\FancyVerbLineautorefname}{Zeile}%
%			\renewcommand*{\theoremautorefname}{Theorem}%
%			\renewcommand*{\appendixautorefname}{Anhang}%
%			\renewcommand*{\equationautorefname}{Gleichung}%        
%			\renewcommand*{\itemautorefname}{Punkt}%
%                }%  
            \providecommand{\subfigureautorefname}{\figureautorefname}%             
    }{\relax}
\makeatother

%*****
% Setup drawing environment
%*****

\usepackage[svgnames,table]{xcolor}
\usepackage{tikz}
\usetikzlibrary{circuits.logic.US,circuits.logic.IEC,circuits.ee.IEC,shapes.geometric}
\usepackage{circuitikz}
\usepackage{tikz-timing} % Timing Diagrams
\usetikztiminglibrary[new={char=Q,reset char=R}]{counters}
\usepackage{rotating} % Rotate an image
\usetikzlibrary{calc} % Do some math in tikz

%*****
% Miscellaneous packages I like
%*****
\usepackage{moreenum} % enumerated lists using first, second, etc.
\usepackage{tcolorbox} % framed paragraphs, like "interest boxes"
\usepackage[normalem]{ulem} % strike-through text
\usepackage{siunitx} % align a table column on a decimal point
\usepackage{fancyvrb} % special chars in verbatim, like underlines
\usepackage{SevenSeg} % draw 7-segment displays

% define specific fancyvrb environments for use in my document
\DefineVerbatimEnvironment
{binDisp}{Verbatim} % Display binary math problems, no line numbering
{fontfamily=courier}

\DefineVerbatimEnvironment
{lineDisp}{Verbatim} % Display verbatim with line numbering
{fontfamily=courier,
 numbers=left}

% Used for wrapping figures
\usepackage{float}
\usepackage{wrapfig}
\restylefloat{figure}

%*****
% Used for merging cells in tables, creating a slash in a cell, 
% adjusting the width of a table to fit the text column
%*****
\usepackage{multicol}
\usepackage{multirow}
\usepackage{slashbox}
\usepackage{adjustbox}

%*****
% I used this to just find the width of a line in cm (so I can create
% appropriately sized graphics). Load the package here and then copy the
% other line in a separate paragraph where you need the size displayed.
%*****
\usepackage{layouts}
%textwidth in cm: \printinunitsof{cm}\prntlen{\textwidth}

% *************************************************
% Glossaries and Acronyms
% Note: Glossaries must be loaded after
% hyperref, babel, polyglossia, inputenc, and fontenc
% *************************************************
\usepackage[style=long,nolist]{glossaries}
\newcommand{\glossname}{Glossary}
\makeglossaries
\loadglsentries{dlGloss}
%\newglossaryentry{LW}{name={Long Word},description={This is just a long word.}}
%\newacronym{wtf}{WTF}{What the Flying Freakin' Flip???}

%*****
% 7. Last calls before the bar closes
%*****

%*****
% Development Stuff
%*****
\listfiles
%\PassOptionsToPackage{l2tabu,orthodox,abort}{nag}
%   \usepackage{nag}
%\PassOptionsToPackage{warning, all}{onlyamsmath}
%   \usepackage{onlyamsmath}

%*****
% Last, but not least...
%*****
\usepackage{classicthesis} 
